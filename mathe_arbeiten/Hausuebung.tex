\documentclass[a4paper,12pt]{article}
\usepackage{graphicx}
\usepackage{amsmath}
\usepackage{amssymb}
\usepackage{amsthm} 
\usepackage{booktabs}
\usepackage{caption} 

%----------------------------------------------------------------------------------------
%	TITLE SECTION
%----------------------------------------------------------------------------------------

\title{	
	\normalfont\normalsize
	\textsc{Technische Universität Wien}\\
    \vspace{10pt}
    \textsc{\large 104.631 Mathematisches Arbeiten für Informatik und Wirtschaftsinformatik}\\
    \rule{\linewidth}{0.5pt}\\
	\vspace{20pt} 
	{\huge N. Hausübung}
}

\author{\Large Name des Studierenden} % Hier den Namen einfügen
\date{\normalsize\today} % Aktuelles Datum

\begin{document}

\maketitle % Titel erzeugen

%----------------------------------------------------------------------------------------
%	ABSCHNITT 1
%----------------------------------------------------------------------------------------

\section{Aufgabe}
% Beispiel: Ein Bild einfügen

%\begin{figure}[ht] 
%	\centering
%	\includegraphics[width=0.5\columnwidth]{Bild.jpg} 
%	\caption{Beispielbild}
%\end{figure}

% Beispiel: Mathematische Umgebungen

\begin{equation}
    \begin{split}
		P(A|B) = \frac{P(B|A)P(A)}{P(B)}
	\end{split}		
\end{equation} 

\begin{equation}
	A = 
	\begin{bmatrix}
		A_{11} & A_{12} \\
		A_{21} & A_{22}
	\end{bmatrix}
\end{equation}

\begin{align}
	B = 
	\begin{bmatrix}
		B_{11} & B_{12} \\
		B_{21} & B_{22}
	\end{bmatrix}
\end{align}

%----------------------------------------------------------------------------------------
%	ABSCHNITT 2
%----------------------------------------------------------------------------------------

\section{Aufgabe}
% Beispiel: Aufzählung

\begin{itemize}
	\item Erster Punkt in einer Liste
		\begin{itemize}
		\item Erster Unterpunkt 
			\begin{itemize}
			\item Tiefer Unterpunkt 1
			\item Tiefer Unterpunkt 2
			\end{itemize}
		\item Zweiter Unterpunkt
		\end{itemize}
	\item Zweiter Punkt in einer Liste 
\end{itemize}

%----------------------------------------------------------------------------------------
%	ABSCHNITT 3
%----------------------------------------------------------------------------------------

\section{Aufgabe}
% Beispiel: Tabelle

\begin{table}[ht] 
	\centering
	\begin{tabular}{l l l}
		\toprule
		\textbf{Pro 50g} & \textbf{Schweinefleisch} & \textbf{Soja} \\
		\midrule
		\textit{Energie} & 760 kJ & 538 kJ \\
		\textit{Protein} & 7.0 g & 9.3 g \\
		\textit{Kohlenhydrate} & 0.0 g & 4.9 g \\
		\textit{Fett} & 16.8 g & 9.1 g \\
		\textit{Natrium} & 0.4 g & 0.4 g \\
		\textit{Ballaststoffe} & 0.0 g & 1.4 g \\
		\bottomrule
	\end{tabular}
	\caption{Nährwerte einer Wurst.}
\end{table}

\end{document}
